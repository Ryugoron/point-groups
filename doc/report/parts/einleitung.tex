\section{Einleitung}

\subsection{Aufgabenstellung}

    \begin{itemize}
        \item Die endlichen Gruppen von 3-dimensionalen orthogonalen Transformationen, die den Ursprung fest lassen, stellen die Symmetrien von 3-dimensionalen Polyedern dar. Solche Polyeder sollen modelliert und graphisch dargestellt werden. Durch Verformungen soll dabei deutlich werden, welche Ecken einander zugeordnet sind.
        \item Die endlichen Gruppen von 4-dimensionalen orthogonalen Transformationen, die den Ursprung fest lassen, stellen die Symmetrien von 4-dimensionalen Polyedern dar. Solche Polyeder lassen sich als dreidimensionale Raumteilungen durch sogenannte Schlegeldiagramme darstellen.
        \item Visualisierung als Schlegeldiagramme;
        \item Anzeigen des Fundamentalbereichs.
    \end{itemize}
    Ziel dieses Softwareprojekts ist die Visualisierung der drei-- und vierdimensionalen Punktgruppen. Punktgruppen aus dem Bereich der euklidischen Geometrie sind endliche Gruppen von orthogonalen Transformationen, die mindestens einen Punkt fest lassen, und die Symmetrien eines drei-- bzw. vierdimensionalen Polyeders darstellen. Diese Polyeder sollen modelliert und grafisch dargestellt werden, sodass durch Interaktion weitere Erkenntnisse darüber gewonnen werden können, welche Ecken einander zugeordnet sind.
    
    Dazu wird anhand einer ausgewählten Punktgruppe ein frei wählbarer Punkte aus dem zugehörigen Fundamentalbereich~\ref{fundamentalbereich} unter den zugehörigen Symmetrien abgebildet. Die Abbildungen werden mittels Quaternionen~\ref{quaternionen}, die zur Darstellung der Rotationen und Spiegelungen verwendet werden, durchgeführt. Anschließend wird die konvexe Hülle der Abbildungen gebildet und visualisiert. Im dreidimensionalen kann das Polyeder der konvexen Hülle direkt grafisch dargestellt werden. Für den vierdimensionalen Fall lassen sich die dreidimensionalen Raumteilungen durch sogenannte Schlegeldiagramme darstellen.
    
    Um dies zu realisieren, muss sowohl der Fundamentalbereich als auch das Schlegeldiagramm berechnet werden. Als Softwarehilfsmittel kommt dazu polymake~\cite{polymake} zum Einsatz, dass eine Reihe von Funktionen im Zusammenhang mit konvexen Polyedern bereitstellt. Für die abschließende grafische Darstellung wird die auf dreidimensionale und mathematische Visualisierung spezialisierte Java Bibliothek jReality~\cite{jreality} verwendet.