\newtheorem{defdef}{Defintion}[section]
\newenvironment{definition}[1][]{\begin{defdef}[#1] \normalfont\hspace*{1mm}}{\hfill $\lrcorner$\end{defdef}\vspace{0.2cm}}

\section{Mathematische Grundlagen}
\subsection{Quaternionen}\label{quaternionen}
Zur einheitlichen Darstellung von drei- und vierdimensionalen Rotationen und Spiegelungen werden
sog. \textbf{Quaternionen} benutzt.
Die Menge der Quaternionen $\mathbb{H}$ bilden einen Erweiterungskörper der komplexen Zahlen $\mathbb{C}$ und besitzen überraschend praktische Eigenschaften für den Einsatz bei geometrischen Anwendungen. Die Algebra der Quaternionen wurde 1843 von Hamilton entwickelt~\cite{hazewinkel2004algebras}.

\begin{definition}[Quaternion]
Es seien $a,b,c,d \in \mathbb{R}$ und $i,j,k$ imaginäre Einheiten für die gilt:
\begin{equation*}\begin{split}
i^2 = j^2 = k^2 = ijk = -1
\end{split}\end{equation*}
....
Dann heißt die Zahl $p = a + bi + cj + dk$ Quaternion. Die Menge der Quaternionen wird mit $\mathbb{H}$ bezeichnet
und ist äquivalent zu rellen Vektorraum $\mathbb{R}^4$.
\end{definition}
....
\begin{definition}[Kojugiertes Quaternion, Norm, Reziprokwert]
...
$q^* := a - bi - cj - dk$ konjugiert
$\| q \| := \sqrt{q q^*}$ norm

\end{definition}

....

\begin{definition}[Einheitsquaternion]
Ein Quaternion $q$ heißt Einheitsquaternion falls $\|q\| = 1$.
\end{definition}

\subsection{Quaternionen und dreidimensionale Symmetrien}
Verwendung von Einheitsquaternionen für Rotationen 
Einheitsquaternionen praktisch.

Eulers Rotationstheorem: Jede Drehung oder Folge von Drehungen um einen festen Punkt kann für eine einfache Drehung um einen Winkel $\theta$ und eine Achse (sog. Eulerachse) durch diesen festen Punkt beschrieben werden. Dafür identizifiert man einen 
Drehwinkel $\theta$ und die Drehachse $\overline{u} = (u_1,u_2,u_3)$.
Dann beschreibt das Quaternion $q = e^{\frac{1}{2}\theta(u_1i + u_2j + u_3k)}$ diese Drehung.
Die Rotationsfunktion wird dann von der Abbildung $[q]: p \mapsto q p q^{-1}$ beschrieben (entspricht der Konjugation von p durch q).
Dadurch, dass bei Einheitsquaternionen gilt $q^{-1} = q^*$, wird auch Rechenaufwand gespart.

\subsection{Quaternionen und vierdimensionale Symmetrien}
....

\subsection{Punktgruppen}
...gibts hier was interessantes?

\subsection{Fundamentalbereiche}\label{fundamentalbereich}
Kurz: Was ist das, wofür braucht man das? Wie kann man es ausrechnen (nicht in polymake oder so, nur rein mathematisch)?
