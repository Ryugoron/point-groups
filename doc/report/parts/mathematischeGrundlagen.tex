\newtheorem{defdef}{Defintion}[section]
\newenvironment{definition}[1][]{\begin{defdef}[#1] \normalfont\hspace*{1mm}}{\hfill $\lrcorner$\end{defdef}\vspace{0.2cm}}

\section{Mathematische Grundlagen}
\subsection{Quaternionen}\label{quaternionen}
Zur einheitlichen Darstellung von drei- und vierdimensionalen Rotationen und Spiegelungen werden
sog. \textbf{Quaternionen} benutzt.
Die Menge der Quaternionen $\mathbb{H}$ bilden einen Erweiterungskörper der komplexen Zahlen $\mathbb{C}$ und besitzen überraschend praktische Eigenschaften für den Einsatz bei geometrischen Anwendungen. Die Algebra der Quaternionen wurde 1843 von Hamilton entwickelt~\cite{hazewinkel2004algebras} und seitdem intensiv erforscht.

\begin{definition}[Quaternion]
Es seien $a,b,c,d \in \mathbb{R}$ und $i,j,k$ imaginäre Einheiten (durch Adjunktion) für die gilt:
\begin{equation*}\begin{split}
i^2 = j^2 = k^2 = ijk = -1
\end{split}\end{equation*}
Dann heißt die Zahl $q = a + bi + cj + dk$ Quaternion.
\end{definition}
Dabei sind die Bezeichnungen $Re(.)$ und $Im(.)$ für den Real- und Imiginärteil eines Quaternions wie üblich definiert.
Die Menge der Quaternionen wird mit $\mathbb{H}$ bezeichnet und ist äquivalent zum reellen Vektorraum $\mathbb{R}^4$.
Man kann also ein Quaternion $q \in \mathbb{H}$ als Vektor auffassen; dies ist speziell für die Interpretation von Quaternionen als Rotationsoperatoren hilfreich.

Wie auch bei den komplexen Zahlen $\mathbb{C}$, können die üblichen Rechenoperationen auf Quaternionen erweitert werden:

\begin{definition}[Kojugiertes Quaternion, Norm, Betrag] $\quad$ \\
Sei $q = a + bi + cj + dk \in \mathbb{H}$. Dann heißt das Quaternion
\begin{equation*}
\overline{q} := a - bi - cj - dk
\end{equation*} das zu $q$ konjugierte Quaternion. \\
Die Norm von $q$ ist gegeben durch $Norm(q) := a^2 + b^2 + c^2 + d^2 = q\overline{q} = \overline{q}q$, der Wert
$\| q \| := \sqrt{Norm(q)} = \sqrt{q \overline{q}}$ wird Betrag (oder Länge) von $q$ genannt.
\end{definition}
\noindent Der Betrag $\|q\|$ eines Quaternions $q = a + bi + cj + dk$ ist identisch mit der (euklidischen) Norm des Vektors $(a,b,c,d) \in \mathbb{R}^4$. Quaternionen mit Norm 1 spielen im Folgenden eine besondere Bedeutung.

\begin{definition}[Einheitsquaternion, Reines Quaternion]
Ein Quaternion $q$ heißt 
\begin{itemize}
\item Einheitsquaternion, falls $\|q\| = 1$
\item Reines Quaternion, falls $Re(q) = 0$
\end{itemize}\vspace{-0.9cm}
\end{definition}
Wegen $\|q\| = 1$ folgt für das Inverse von Einheitsquaternionen, dass $q^{-1} = \overline{q}$. Das ist sehr praktisch,
da somit eine effiziente Berechnungsvorschrift für das Invertieren von Einheitsquaternionen vorliegt.

\subsection{Quaternionen und dreidimensionale Symmetrien\label{ssec:quad3sym}}
Die Benutzung von Quaternionen für dreidimensionale Symmetrien ist zwar nicht so intuitiv wie z.B. die Benutzung von Matrizen,
erlaubt einem jedoch eine sehr elegante und schlanke Darstellung von Rotationen~\cite{conway2003}~\cite{du1964}: 

Nach Eulers Rotationstheorem kann jede Drehung oder Folge von Drehungen um einen festen Punkt durch eine einzige Drehung beschrieben werden.
Hierfür benötigen wir zum einen den Drehwinkel $\theta$ und eine Achse (die sog. Euler-Achse) die durch eben den festen Punkt geht.
Der Drehwinkel $\theta$ ist dabei, wie üblich, ein Skalar; die Euler-Achse $\overrightarrow{u}$ wird durch einen dreidimensionalen Einheitsvektor $\overrightarrow{u} = (u_1,u_2,u_3)$ dargestellt. Diese vier Werte der Rotation (also $\theta, u_1, u_2, u_3$) können durch eine simple
Berechnungsvorschrift in einen (Einheits-)Quaternion kodiert und dann anschließend durch simple Rechenoperationen
angewendet werden.

Hierfür wird die Eulerachse $\overrightarrow{u}$ als Quaternion interpretiert; es ergibt sich das reine Einheitsquaternion
$u = u_1 i + u_2 j + u_3 k$. Das Rotationsquaternion $q$ kann dann nach Eulers Formel wie folgt berechnet werden:
\begin{equation*}
q = e^{\frac{1}{2} \theta \left( u_1 i + u_2 j + u_3 k \right)} = \cos \frac{1}{2} \theta + \left( u_1 i + u_2 j + u_3 k \right) \sin \frac{1}{2} \theta
\end{equation*}

\noindent Wollen wir nun einen Punkt $(p_x, p_y, p_z) \in \mathbb{R}^3$ gemäß $q$ rotieren, so wird der Punkt wiederum als reines Quaternion
$p = p_x i + p_y j + p_z k$ interpretiert und schlicht die Konjugation mit $q$ berechnet:
Die durch $q$ induzierte Rotationsfunktion $[.]$ ergibt sich dann also durch die Abbildung
\begin{equation*}
[q]: p \mapsto q p q^{-1}
\end{equation*}
Diese Berechnung ist mit zwei Quaternionen-Multiplikationen nicht besonders aufwändig.
Da es sich bei $p$ um ein Einheitsquaternion handelt, gilt außerdem $q^{-1} = \overline{q}$ -- dieser Berechnungsaufwand
kann also beinahe vernachlässigt werden. \\

\noindent Für Spiegelungen können Quaternionen ebenso elegant benutzt werden~\cite{1946}: Soll ein Punkt  $(p_x, p_y, p_z) \in \mathbb{R}^3$
an einer Ebene $\Pi: vn = 0$ durch den Ursprung mit Normalenvektor $n$ gespielt werden, so wird dies durch die Funktionsvorschrift $].[$ mit
\begin{equation*}
]n[: p \mapsto n p n
\end{equation*}
beschrieben, wobei sich $p$ wieder durch Interpretation von $(p_x, p_y, p_z)$ als reines Quaternion ergibt.

\subsection{Quaternionen und vierdimensionale Symmetrien}
%%%%%%%%%%%%%%%%%  4D-case
Quaternionen eignen sich besonders für den vierdimensionalen Raum. Vierdimensionale Rotationen können nach ~\cite{conway2003} auf zwei dreidimensionale Rotationen reduziert werden.
\noindent Wollen wir nun einen Punkt $p$ \in \mathbb{R}^4$ rotieren, so wird die Rotation durch die beiden Quaternionen $l,r$ repräsentiert.
Die durch $l,r$ induzierte Rotationsfunktion $[.]$ ergibt sich dann also durch die Abbildung
\begin{equation*}
[l,r]: p \mapsto l p r
\end{equation*}
Im Gegensatz zu Rotationen muss nur ein Quaternion mehr gespeichert werden,  aber der Berechnungsaufwand ist gleich.\\
Für Spiegelung werden jede Rotationsgruppe mit Hilfe eines Operators \todo{Welche?}erweitert. Alle Gruppenelemente werden mit diesem multipliziert. Dadurch bleiben die Untergruppenbeziehungen aus den Rotationsgruppen unter Beachtung des Operators erhalten.
Die Gruppen im vierdiemensionalen Fall sind nur mittels Generatoren in ~\cite{conway2003} beschrieben. Die Gruppegrößen variieren zwischen \todo{Wie groß sind die Gruppe?}. Eine explizierte Aufzählung exisiert nicht. 

\subsection{Punktgruppen}
ALEX SCHREIBT HIER KURZE SAETZE...gibts hier was interessantes?

\subsection{Fundamentalbereiche}\label{fundamentalbereich}

Wie bereits erwähnt wurde, ist eine Punktgruppe die Gruppenwirkung einer Symmetriegruppe auf ein gegebenes Element im zugrunde liegendem Raum der Symmetrieoperationen.
Für eine Gruppenwirkung können wir zunächst die Verallgemeinerung der Punktgruppe definiere.

\begin{definition}[Gruppenwirkung]\label{fundamentalbereich:wirking} \mbox{}\\
 Eine (Links-)Wirkung einer Gruppe $(G, \star)$ auf eine Menge $X$ ist eine Funktion
   $$
      \rhd \, : \, G \times X \longrightarrow X
   $$
   mit den Eigenschaften
   \begin{itemize}
      \item $(g \star h) \rhd x = g \rhd (h \rhd x)$ für alle $g,h \in G$ und $x \in X$ und
      \item $e \rhd x = x$ für alle $x \in X$ und $e$ neutrales Element von $G$.
   \end{itemize}
\end{definition}

\begin{definition}[Orbit] \label{fundamentalbereich:orbit} \mbox{}\\
  Für eine Gruppe $(G, \star)$ mit Gruppenwirkung $\rhd$ auf $X$ ist für ein Element $x \in X$ die \emph{Bahn von $x$ bezüglich $G$} definiert als
   $$
      G \rhd x := \left\{ g \rhd x \, | \, g \in G \right\}.
   $$
\end{definition}

Über die Orbits können wir eine Äquivalenzrelation beschreiben, mit $x_1$ ist äquivalent zu $x_2$, geschrieben $x_1 \sim x_2$, genau dann, wenn $x_2 \in G \rhd x_1$ und $G \setminus X$ die Menge der Repräsentanten ist.\\

Nun bildet $G \setminus X$ schon annähernd einen Fundamentalbereich, falls $G$ eine Symmetriegruppe ist. Symmetrien auf $\mathbb{R}^n$ sind Isometrien und folglich bleiben die Länge der Vektoren immer erhalten. Aufgrund dieser Tatsache interessieren uns bei der Betrachtung nur Vektoren gleicher Länge. Wir definieren daher den Fundamentalbereich wie folgt.\\

\begin{definition}[Fundamentalbereich]\label{fundamentalbereich:def} \mbox{}\\
   Sei $G \subseteq O(n)$ eine Symmetriegruppe auf $\mathbb{R}^n$.

   \noindent Dann ist ein \emph{Fundamentalbereich von $G$} ein Representatensystem $G \setminus S^{n-1}$.
\end{definition} 

Für die Darstellung der Fundamentalbereiche, wählen wir nur spezielle, zusammenhängende Repräsentanten.

\begin{definition}[Voronoi--Fundamentalbereich]\label{fundamentalbereich:voronoi} \mbox{}\\
   Sei $O \subset 2^{S^{n-1}}$ die Menge der Orbits der Gruppenwirkung von $G$ auf $S^{n-1}$ und $x \in S^{n-1}$ ein ausgezeichneter Punkt.\\

   Dann ist der Voronoi--Fundamentalbereich die Menge
   $$
      VF(x) := \left\{ \underset{y\in o}{\text{argmin}} \, d(x,y) \, |  o \in O\right\}.
   $$
   Damit ist die Menge der $VF(y)$ mit $y \in G \rhd x$ eine Voronoidiagram, da wir jeweils die Punkte mit minimalem Abstand genommen haben.
\end{definition}


