\section{Projektorganisation}
%    \begin{itemize}
%        \item spline Pad für Aufgabenverteilung, Statusübersicht
%        \item Wöchentliche Meetings
%        \item GitHub zur Versionskontrolle, Wiki, Doku
%        \item Mailinglist zur Kommunikation abseits der wöchentlichen Treffen
%    \end{itemize}
    
Bei einer eher geringen Gruppengröße von 6 Personen, haben wir uns dafür entschieden keine aufwändige Projektstruktur anzulegen. Kern unserer Projektorganisation war ein wöchentliches Treffen am Institut. Dabei stellte jeder seine Ergebnisse der letzten Woche vor. Dazu gehörten auch eventuelle Schwierigkeiten, die anschließend in der Gruppe diskutiert wurden. Die daraus entstandene Lösungsansätze wurde dem Aufgabenkatalog der aktuellen Woche hinzugefügt.\\
Die Ergebnisse dieser Besprechungen und die Aufgaben für die nächste Woche wurden in einem Spline--Pad festgehalten (\url{http://pad.spline.de}). So konnte jederzeit der aktuelle Stand und anstehende Aufgaben nachgelesen werden. Auch Designentscheidungen ließen sich anhand dieser Protokolle nachvollziehen.\\
Um parallel am selben Quellcode arbeiten zu können, haben wir ein Projekt auf GitHub (\url{https://github.com}) erstellt. Hier wurde unser Projekt gehostet und mit Git verwaltet. Wir haben die Entwicklung in unterschiedlichen Branches durchgeführt, sodass jede neue Entwicklung zuerst in einem eigenen Branch publiziert und getestet wurde. Anschließend kamen ausreichend getestete Neuerungen in den Master--Branch, der immer die neueste, stabile Version des Projekts enthielt. Das heißt, der Master--Branch enthielt immer eine lauffähige oder zumindest kompilierende Version des Projekts und war somit der Ausgangspunkt für alle Weiterentwicklungen.\\
Zur Dokumentation haben wir auf GitHub ein Wiki geführt. Dies haben wir hauptsächlich dafür genutzt, die notwendigen mathematischen Berechnungen zur Realisierung des Projekts zu erfassen und den Aufbau des Systems zu dokumentieren. Auch Quellen, aus denen wir die Berechnungen abgeleitet haben, finden sich hier. Zusätzlich zu dieser Form der Dokumentation war jeder angehalten, seinen Quelltext ausreichend zu kommentieren --- hauptsächlich mit Javadoc ---, sodass dieser auch von anderen Gruppenmitgliedern verstanden und erweitert werden konnte.\\
Des Weiteren haben wir uns eine Mailingliste bei Spline eingerichtet (\url{http://lists.spline.de/}), die zur Kommunikation abseits der wöchentlichen Treffen diente. Hierüber konnten Fragen gestellt und diskutiert werden, die zwischen den wöchentlichen Treffen aufkamen und deren Klärung für ein Weiterarbeiten notwendig waren. Außerdem erhielt dadurch jeder einen Eindruck, welche Probleme die Anderen gerade hatten. Abweichende Termine für das nächste Treffen haben wir über die Mailingliste ebenfalls abgestimmt und angekündigt.
\todo{Irgendwie ist das alles sehr nichtssagend. Aber viel mehr kann man dazu auch nicht schreiben, oder?}