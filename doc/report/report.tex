\documentclass[a4paper]{scrartcl}
\usepackage[utf8]{inputenc}  
\usepackage[T1]{fontenc}
\usepackage{lmodern}           
\usepackage[ngerman]{babel}

%\usepackage[perpage,bottom]{footmisc} % Footnote configuration.

\usepackage{amsmath} % More math.
\usepackage{amssymb} % More symbols.
\usepackage{fancyhdr} % For header and footer format
%\usepackage{fancyref} % For fancy references
\usepackage{listings} % For code listings
\usepackage{booktabs} % For tables
\usepackage[perpage,bottom]{footmisc} % Footnote configuration
\usepackage{graphicx} % For includegraphics
\usepackage[page,titletoc]{appendix} % For appendix in toc
\usepackage{color} % For colors, not sure if I truly need it anymore
\usepackage{caption} % For nicer listing captions
\usepackage{xspace} % For horizontal space.
\usepackage{pst-gantt} % For GANTT chart.
\usepackage{subfigure} % For placing figures next to each other.
\usepackage{wrapfig} % for wraping text around a image
\usepackage{eso-pic} % for \addtoshipoutpicture
\usepackage{todonotes} % mark todos

% Keine Ahnung warum, aber "`bla"' funktioniert bei mir nicht.
\newcommand{\enquote}[1]{\glqq{}#1\grqq{}}

% Big O
\newcommand{\bigO}{\ensuremath{\mathcal{O}}}

% Convex Hull
\newcommand{\CH}{\ensuremath{\mathcal{CH}}}

% Lesson learned.
\newenvironment{lessonlearned}{\vspace{0.3cm}\begin{quotation}}{\end{quotation}\vspace{0.1cm}}

% Hyperref macht Links hinter die Sections und manipuliert PDF-Metadaten:
\usepackage[
colorlinks=false,
pdfborder={0 0 0},
pdftitle={Visualisierung von drei- und vierdimensionalen Punktgruppen},
pdfsubject={Abschlussbericht zum Softwareprojekt \enquote{Visualisierung der 3- und 4-dimensionalen Punktgruppen}},
pdfauthor={Marcel Ehrhardt, Nadja Scharf, Alexander Steen, Simon Tippenhauer, Oliver Wiese, Max Wisniewski}
%pdfkeywords={}
]{hyperref}
\usepackage{breakurl}

% Pimp my listings environment.
% Credits go to stackoverflow:
% http://stackoverflow.com/questions/741985/latex-source-code-listing-like-in-professional-books
\lstset{
         basicstyle=\footnotesize\ttfamily, % Standardschrift
         numberstyle=\tiny,          % Stil der Zeilennummern
         numbers=left,
         stepnumber=1,
         numbersep=7pt,              % Abstand der Nummern zum Text
         keywordstyle=\bfseries\ttfamily,
         tabsize=2,                  % Groesse von Tabs
         extendedchars=true,         %
         breaklines=true,            % Zeilen werden Umgebrochen       
         showspaces=false,           % Leerzeichen anzeigen ?
         showtabs=false,             % Tabs anzeigen ?
         %frame=b,                   % Linie unten
         breakatwhitespace=true,
         xleftmargin=17pt,
         framexleftmargin=17pt,
         framexrightmargin=5pt,
         framexbottommargin=4pt,
         showstringspaces=false      % Leerzeichen in Strings anzeigen ?      
 }

% Pimp my captions.
\DeclareCaptionFont{white}{\color{white}}
\DeclareCaptionFormat{listing}{\colorbox[cmyk]{0.43, 0.35, 0.35,0.01}{\parbox{\textwidth}{\hspace{15pt}#1#2#3}}}
\captionsetup[lstlisting]{format=listing,labelfont=white,textfont=white, singlelinecheck=false, margin=0pt, font={bf}}

% Listing environment for C
\lstnewenvironment{code}[1][]
  {\noindent\minipage{\linewidth} 
   \lstset{language=C,#1}}
  {\endminipage}

% Konfiguration der Titelseite:
\newcommand\fulogo{%
   \put(0,0){%
      \parbox[b][\paperheight]{\paperwidth}{%
         \vspace{1.5cm}
         \begin{center}
	         \includegraphics[width=0.7\textwidth]{img/fulogo-eps-converted-to.pdf}%
         \end{center}
         \vfill{}
      }
   }
}
\title{
{\vspace{1.4cm}\normalsize Softwareprojekt über Anwendung effizienter Algorithmen\\Dozent: Prof. Dr. Günter Rote\\Freie Universität Berlin\\Wintersemester 2013/14}
\\[4ex] 
{\Large Abschlussbericht zum Softwareprojekt\\ \enquote{Visualisierung der 3-- und 4--dimensionalen Punktgruppen}}
\author{Marcel Ehrhardt \and Nadja Scharf \and Alexander Steen \and Simon Tippenhauer \and Oliver Wiese \and Max Wisniewski}
\date{
\vspace{1.0cm}
\today{}
}
}

\begin{document}
\begin{titlepage}
\AddToShipoutPicture*{\fulogo}
\pagenumbering{alph}
\maketitle
\thispagestyle{empty}
\vfill{}
\end{titlepage}

\pagestyle{empty}
\pagenumbering{roman}
%\include{parts/abstract}
\tableofcontents
\clearpage

\pagenumbering{arabic}
\pagestyle{fancy}
\lhead[]{}
\rhead[\nouppercase{\rightmark}]{\nouppercase{\rightmark}}
\setcounter{page}{1}

% Hier hin unsere Kapitel.

\section{Einleitung}

\subsection{Aufgabenstellung}

    \begin{itemize}
        \item Die endlichen Gruppen von 3-dimensionalen orthogonalen Transformationen, die den Ursprung fest lassen, stellen die Symmetrien von 3-dimensionalen Polyedern dar. Solche Polyeder sollen modelliert und graphisch dargestellt werden. Durch Verformungen soll dabei deutlich werden, welche Ecken einander zugeordnet sind.
        \item Die endlichen Gruppen von 4-dimensionalen orthogonalen Transformationen, die den Ursprung fest lassen, stellen die Symmetrien von 4-dimensionalen Polyedern dar. Solche Polyeder lassen sich als dreidimensionale Raumteilungen durch sogenannte Schlegeldiagramme darstellen.
        \item Visualisierung als Schlegeldiagramme;
        \item Anzeigen des Fundamentalbereichs.
    \end{itemize}
    Ziel dieses Softwareprojekts ist die Visualisierung der drei-- und vierdimensionalen Punktgruppen. Punktgruppen aus dem Bereich der euklidischen Geometrie sind endliche Gruppen von orthogonalen Transformationen, die mindestens einen Punkt fest lassen, und die Symmetrien eines drei-- bzw. vierdimensionalen Polyeders darstellen. Diese Polyeder sollen modelliert und grafisch dargestellt werden, sodass durch Interaktion weitere Erkenntnisse darüber gewonnen werden können, welche Ecken einander zugeordnet sind.
    
    Dazu wird anhand einer ausgewählten Punktgruppe ein frei wählbarer Punkte aus dem zugehörigen Fundamentalbereich~\ref{fundamentalbereich} unter den zugehörigen Symmetrien abgebildet. Die Abbildungen werden mittels Quaternionen~\ref{quaternionen}, die zur Darstellung der Rotationen und Spiegelungen verwendet werden, durchgeführt. Anschließend wird die konvexe Hülle der Abbildungen gebildet und visualisiert. Im dreidimensionalen kann das Polyeder der konvexen Hülle direkt grafisch dargestellt werden. Für den vierdimensionalen Fall lassen sich die dreidimensionalen Raumteilungen durch sogenannte Schlegeldiagramme darstellen.
    
    Um dies zu realisieren, muss sowohl der Fundamentalbereich als auch das Schlegeldiagramm berechnet werden. Als Softwarehilfsmittel kommt dazu polymake~\cite{polymake} zum Einsatz, dass eine Reihe von Funktionen im Zusammenhang mit konvexen Polyedern bereitstellt. Für die abschließende grafische Darstellung wird die auf dreidimensionale und mathematische Visualisierung spezialisierte Java Bibliothek jReality~\cite{jreality} verwendet.
\section{Projektorganisation}
%    \begin{itemize}
%        \item spline Pad für Aufgabenverteilung, Statusübersicht
%        \item Wöchentliche Meetings
%        \item GitHub zur Versionskontrolle, Wiki, Doku
%        \item Mailinglist zur Kommunikation abseits der wöchentlichen Treffen
%    \end{itemize}
    
Bei einer eher geringen Gruppengröße von 6 Personen, haben wir uns dafür entschieden keine aufwändige Projektstruktur anzulegen. Kern unserer Projektorganisation war ein wöchentliches Treffen am Institut. Dabei stellte jeder seine Ergebnisse der letzten Woche vor. Dazu gehörten auch eventuelle Schwierigkeiten, die anschließend in der Gruppe diskutiert wurden. Die daraus entstandenen Lösungsansätze wurden dem Aufgabenkatalog der aktuellen Woche hinzugefügt.\\
Die Ergebnisse dieser Besprechungen und die Aufgaben für die nächste Woche wurden in einem Spline--Pad festgehalten (\url{http://pad.spline.de}). So konnte jederzeit der aktuelle Stand und anstehende Aufgaben nachgelesen werden. Auch Designentscheidungen ließen sich anhand dieser Protokolle nachvollziehen.\\
Um parallel am selben Quellcode arbeiten zu können, haben wir ein Projekt auf GitHub (\url{https://github.com}) erstellt. Hier wurde unser Projekt gehostet und mit Git verwaltet. Wir haben die Entwicklung in unterschiedlichen Branches durchgeführt, sodass jede neue Entwicklung zuerst in einem eigenen Branch publiziert und getestet wurde. Anschließend kamen ausreichend getestete Neuerungen in den Master--Branch, der immer die neueste, stabile Version des Projekts enthielt. Das heißt, der Master--Branch enthielt immer eine lauffähige oder zumindest kompilierende Version des Projekts und war somit der Ausgangspunkt für alle Weiterentwicklungen.\\
Zur Dokumentation haben wir auf GitHub ein Wiki geführt. Dies haben wir hauptsächlich dafür genutzt, die notwendigen mathematischen Berechnungen zur Realisierung des Projekts zu erfassen und den Aufbau des Systems zu dokumentieren. Auch Quellen, aus denen wir die Berechnungen abgeleitet haben, finden sich hier. Zusätzlich zu dieser Form der Dokumentation war jeder angehalten, seinen Quelltext ausreichend zu kommentieren --- hauptsächlich mit Javadoc ---, sodass dieser auch von anderen Gruppenmitgliedern verstanden und erweitert werden konnte.\\
Des Weiteren haben wir uns eine Mailingliste bei Spline eingerichtet (\url{http://lists.spline.de/}), die zur Kommunikation abseits der wöchentlichen Treffen diente. Hierüber konnten Fragen gestellt und diskutiert werden, die zwischen den wöchentlichen Treffen aufkamen und deren Klärung für ein Weiterarbeiten notwendig waren. Außerdem erhielt dadurch jeder einen Eindruck, welche Probleme die Anderen gerade hatten. Abweichende Termine für das nächste Treffen haben wir über die Mailingliste ebenfalls abgestimmt und angekündigt.


\newtheorem{defdef}{Defintion}[section]
\newenvironment{definition}[1][]{\begin{defdef}[#1] \normalfont\hspace*{1mm}}{\hfill $\lrcorner$\end{defdef}\vspace{0.2cm}}

\section{Mathematische Grundlagen}
\subsection{Quaternionen}\label{quaternionen}
Zur einheitlichen Darstellung von drei- und vierdimensionalen Rotationen und Spiegelungen werden
sog. \textbf{Quaternionen} benutzt.
Die Menge der Quaternionen $\mathbb{H}$ bilden einen Erweiterungskörper der komplexen Zahlen $\mathbb{C}$ und besitzen überraschend praktische Eigenschaften für den Einsatz bei geometrischen Anwendungen. Die Algebra der Quaternionen wurde 1843 von Hamilton entwickelt~\cite{hazewinkel2004algebras}.

\begin{definition}[Quaternion]
Es seien $a,b,c,d \in \mathbb{R}$ und $i,j,k$ imaginäre Einheiten für die gilt:
\begin{equation*}\begin{split}
i^2 = j^2 = k^2 = ijk = -1
\end{split}\end{equation*}
....
Dann heißt die Zahl $p = a + bi + cj + dk$ Quaternion. Die Menge der Quaternionen wird mit $\mathbb{H}$ bezeichnet
und ist äquivalent zu rellen Vektorraum $\mathbb{R}^4$.
\end{definition}
....
\begin{definition}[Kojugiertes Quaternion, Norm, Reziprokwert]
...
$q^* := a - bi - cj - dk$ konjugiert
$\| q \| := \sqrt{q q^*}$ norm

\end{definition}

....

\begin{definition}[Einheitsquaternion]
Ein Quaternion $q$ heißt Einheitsquaternion falls $\|q\| = 1$.
\end{definition}

\subsection{Quaternionen und dreidimensionale Symmetrien}
Verwendung von Einheitsquaternionen für Rotationen 
Einheitsquaternionen praktisch.

Eulers Rotationstheorem: Jede Drehung oder Folge von Drehungen um einen festen Punkt kann für eine einfache Drehung um einen Winkel $\theta$ und eine Achse (sog. Eulerachse) durch diesen festen Punkt beschrieben werden. Dafür identizifiert man einen 
Drehwinkel $\theta$ und die Drehachse $\overline{u} = (u_1,u_2,u_3)$.
Dann beschreibt das Quaternion $q = e^{\frac{1}{2}\theta(u_1i + u_2j + u_3k)}$ diese Drehung.
Die Rotationsfunktion wird dann von der Abbildung $[q]: p \mapsto q p q^{-1}$ beschrieben (entspricht der Konjugation von p durch q).
Dadurch, dass bei Einheitsquaternionen gilt $q^{-1} = q^*$, wird auch Rechenaufwand gespart.

\subsection{Quaternionen und vierdimensionale Symmetrien}
....

\subsection{Punktgruppen}
...gibts hier was interessantes?

\subsection{Fundamentalbereiche}\label{fundamentalbereich}
Kurz: Was ist das, wofür braucht man das? Wie kann man es ausrechnen (nicht in polymake oder so, nur rein mathematisch)?

\section{Umsetzung}

\subsection{Verwendete Softwarehilfsmittel}
    \begin{itemize}
        \item jReality~\cite{jreality}
        \item polymake~\cite{polymake}
        \item maven~\cite{maven}
        \item jUnit
    \end{itemize}
    
\subsection{Softwarearchitektur}
    \begin{itemize}
        \item Eventsystem (Eventstruktur)
        \item Pipeline (Symmetriegruppe wählen, Fundamentalbereich berechnen und anzeigen, Punkt im Fundamentalbereich auswählen, Punkt unter Symmetrien abbilden, konvexe Hülle Berechnen, im 3--dimensionalen konvexe Hülle anzeigen/im 4--dimensionalen Schlegeldiagram berechnen, anzeigen)
        \item GUI
    \end{itemize}
    \subsubsection{Eventssystem}
        \subsubsection*{Events}
            Events sind konkrete Reaktionen auf bestimmte Ereignisse und enthalten entsprechende Kontextinformationen. Alle Typen von Events erben von der abstrakten Klasse Event<H extends EventHandler>. Klassen, die auf bestimmte Event-Typen reagieren wollen, implementieren das zum Event zugeordnete Interface EventHandler.

            Jede Klasse, die Events abfeuern möchte, muss einen Verweis auf den EventDispatcher besitzen. Ein Singleton von diesem Dispatcher kann mittels EventDispatcher.get() eingeholt werden. Soll ein konkretes Event e gefeuert werden, wird dies via dispatcher.fireEvent(e) erledigt. Alle Klassen, die sich vor dem Feuern eines Events via dispatcher.addHandler(eventType, handler) registriert haben, werden die Events vom Typ eventType erhalten. Der Event-Typ sollte, via Konvention, durch KonkretesEvent.TYPE gegeben sein (wobei KonkretesEvent eine Implementieren von Event ist).
        \subsubsection*{Eventimplementierung}
            Event-Typen werden durch Anlegen einer Event-Klasse und einer EventHandler-Klasse hinzugefügt.

            Das EventHandler-Template ist
            
            \begin{code}
                import pointGroups.gui.event.EventHandler;

                public interface ConcreteHandler
                    extends EventHandler
                {
                    public void onConcreteEvent(final ConcreteEvent event);
                }
            \end{code}

            Das Event-Typ-Template ist

            \begin{code}            
                import pointGroups.gui.event.Event;

                public class ConcreteEvent
                    extends Event<ConcreteHandler>
                {
                    public final static Class<ConcreteHandler> TYPE =
                        ConcreteHandler.class;

                    @Override
                    public final Class<ConcreteHandler> getType() {
                        return TYPE;
                    }

                    @Override
                    protected void dispatch(final ConcreteHandler handler) {
                        handler.onConcreteEvent(this);
                    }
                } 
            \end{code}
            
            wobei Concrete durch konkrete Namen ersetzt werden kann.
            
        \subsubsection*{Implementierte Events}
            Eventuelle Auflistung aller oder nur einzelner.
                
\subsection{Fancy name}
    \begin{itemize}
        \item Symmetriegruppen (Berechnung mittels Generatoren, hart kodiert, Darstellung/Repräsentation)
        \item Fundamentalbereich (Berechnung, Darstellung/Repräsentation)
    \end{itemize}
    
    \subsubsection{Symmetriegruppen}
    \subsubsection{Fundamentalbereich}
         In Abschnitt \ref{fundamentalbereich} wurde in Definition \ref{fundamentalbereich:voronoi} der Voronoi-Fundamentalbereich beschrieben. Wir wollen eine dieser Zellen
         nehmen und Visualisieren. Da wir einen Abschnitt der $S^3$ nicht leicht visualsieren können, benötigen wir zunächst eine Projektion das Vornoi-Fundamentalbereiches auf
         eine $\mathbb{R}^3$ Ebene.
        \subsubsection*{Idee}
            Wir lassen zunächst die Gruppe auf ein Ausgezeichnetes Element $x$ wirken.

            Als erstes Berechnen wir die Voronoi - Zelle für $x$ bezüglich des Orbits von $x$. Alle Punkte die dort drinnen liegen sind Elemente von $S^3$. Da die Gruppe symmet          risch ist, können wir nicht zwei Elemente aus dem selben Orbit haben, da der zweite Punkt näher an einem anderen Element aus $G \rhd x$ liegen müsste.

         Das einzig wichtige an diesem Punkt ist, dass $x$ nicht auf einer Rotationasachse oder Spiegelebene befindet, da sonst die Voronoizellen zu groß ausfallen.

         Haben wir diese Zelle, nehmen wir eine beliebige Ebene, die tangential auf dem Kugelsegment ist und projezieren darauf. Die spezielle Ebene, die wir wählen
         ist die Ebene $h \, : \, [t - x] \cdot x = 0$ in Normalendarstellung, da wir wissen, dass $x$ auf der Kugeloberfläche im Kreissegment liegt und $x$ auch ein
          Normalenvektor ist.
            
        \subsubsection*{Umsetzung}
         Eine Voronoizelle von $x$ für eine Menge von Sites $S$ ist definiert als 
         $$ VC(x) = \bigcap_{s \in S \setminus \{ x \}} h^+_{x,s}$$
         Wobei $h^+_{x,s}$ der Halbraum ist, der rechts von der Hyperebene liegt, die zu $x$ und $y$ in jedem Punkt den selben Abstand hat.\\

         Polymake hat die Möglichkeit ein Polytope, das die Voronoizelle ja ist, gerade über den Schnitt von Halbräumen zu definieren. Dies geht über

         \begin{code}
            new Polytope(INEQUALITIES=>\$hyperplanes)
         \end{code}

         wobei \$hyperplanes eine Menge von Hyperebenen ist. Falls ein affines Polytope vorliegt, wie in unserem Fall in der Ebene $h$, die wie im letzten Abschnitt
         definiert ist, können wir auch dies mit in die Erzeugung eingeben mittels

         \begin{code}
            new Polytope(INEQUALITIES=>\$hyperplanes, EQUATIONS=>{h})
         \end{code}

         und haben so $VC(x)$ berechnet. Da die konkrete Berechnung über den Schnitt von allen Hyperebenen zu lange dauert, filtern wir die in frage kommenden Hyperebenen vor
         über die Nachbern des Voronoi Diagrams. Dummerweise konnten wir nicht ermitteln, wie man in Polymake sich direkt eine Voronoizelle ausgeben lassen kann und
         haben daher diesen Umweg gewählt.\\

         Haben wir nun das Polytope in der Ebene, müssen wir zur Darstellung in $n-1$ Dimensionen noch eine geeignete Basis berechnen, so dass wir
         die Eckpunkte des Polytopes durch $n-1$ Koordinaten darstellen können. Wir wissen schon, dass $x$ senkrecht auf der Ebene steht, also für unsere Darstellung 
         unerheblich ist. Wir berechnen eine Orthonormalbasis für die Ebene mit $x$ als erstem Basisvektor. Nun können wir leicht eine Basiswechselmatrix angeben,
         da wir ja die Bilder der neuen Basis in der alten kennen. Darüber hinaus ist diese Matrix orthogonal -- da die Spalten ja eine orthogonal Basis waren -- und
         kann daher durch transponieren invertiert werden.

         Wir haben nun also zwei Matrizen um beide Darstellungen in einander umzurechnen. In der neuen Basis können wir nun die erste Komponente weg lassen,
         da diese nach Projektion nur genullt wird und haben so eine Darstellung in $n-1$ Dimensionen.

         Beim zurückrechnen müssen wir uns nur erinnern, dass wir uns in einer affinen Ebene mit Stützvektor $x$ befinden.

        \subsubsection*{Berechnung}

         Die Hyperebenen Definition in Polymake hat die Darstellung
         $$
            h \, : \, [a_0, ..., a_n] \rightsquigarrow a_0 + a_1 x_1 + \cdots + a_n x_n = 0
         $$
         und Halbräume ebenso mit $\geq 0$.\\

         Die Halbräume für unser Polytope genügen der Gleichung
         $$
            t \cdot (x - s) \geq 0 \Leftrightarrow 0 + (x_1-s_1)\cdot t_1 + \cdots + (x_n -s_n)t_n \geq 0
         $$
         wobei $x$ der gewählte Vektor war und $s \not x$ aus dem Orbit von $x$ stammt.

         Damit können wir alle Halbräume aus dem Orbit berechnen. Die affine Ebene genügt der Gleichung
         $$
            (t - x) \cdot x = 0 \Leftrightarrow - \left( \sum_{i=1}^n x_i^2 \right) + t_1 x_1 + \cdots t_n x_n = 0
         $$
         und kann auch leicht erstellt werden. Um an unser Polytope heran zu kommen, benutzen wir das folgende Polymake-Script

         \begin{code}
            my \$poly = new Polytope(INEQUALITIES=>\$hyperplanes, EQUATIONS=>\$affine);
            print \$poly->VERTICES;
            print \$poly->EDGES;
            print \$poly->FACETS;
         \end{code}

         Mit diesen Ergebnissen können wir nun die Basiswechselmatrizen berechnen und so die Vertices umrechnen. Das Object Fundamental ist somit
         eine Sammlung dieser Eigenschaften.

         \begin{code}
            public interface Fundamental {
               public double[][] getVertices();
               
               public Edge<Integer,Integer>[] getEdges();

               public double[] revertPoint(double[] point);

               public boolean inFundamental(double[] point);
            }
         \end{code}
         
         Die ersten beiden Funktionen sind selbst erklärend. Die dritte methode \emph{revertPoint} nimmt einen Punkt in $\mathbb{R}^{n-1}$ und
         liftet ihn auf die Oberfläche der $S^{n-1}$ zurück. Die letzte Methode überprüft, ob ein angegebener Punkt überhaupt in der Voronoizelle lag.
         Diese Methode wird bei der anzeige beötigt um bei der Punktauswahl keine Punkte außerhalb das Polytops zuzulassen.\\

         Das ganze ist ein interface, da wir als Fallback-case, falls die Berechnung das Fundamentalbereichs einmal fehl schlägt immer noch 
         den Bereich $S^{n-2}$ benutzen können. Dies ist kein Fundamentalbereich, da wir alle Orbite mehrfach treffen, allerdings werden wir außer im Falle
         der Identität wirklich jeden Orbit mindestens einmal treffen.

\section{Fazit}
In diesem Softwareprojekt wurde eine Java-Applikation entwickelt, die die Punktgruppen der drei- und vierdimensionalen
Polyeder-Symmetrien grafisch darstellt. Zu diesem Zweck wurde \emph{polymake}~\cite{polymake}, eine Berechnungsbibliothek für
konvexe Polytope, als externe Komponente an die Applikation angebunden. Zur Darstellung der Symmetrien wurde dann die Computergrafik-Bibliothek
\emph{jReality}~\cite{jreality} verwendet.
Es wurden sowohl drei- als auch vierdimensionale Punktgruppen samt einiger ihrer Untergruppen implementiert.

\paragraph{Erreichung von Zielen.}
Das Visualisieren der Punktgruppen als dreidimensionale Polyeder bzw. als Schlegeldiagramme wurde erfolgreich umgesetzt. Ebenso wird für jede Punktgruppe ein Fundamentalbereich angezeigt, aus dem ein Punkt zur Berechnung gewählt werden kann. 
Einzig die Zuordnung der Kanten bei Verformungen ist problematisch, da die zur Anzeige benutze Bibliothek kaum Kontrolle über
einen glatten Darstellungsübergang erlaubt.

\paragraph{Einschätzung von Problemen.}
Einige Probleme bereitete uns der Einsatz von Drittsoftware (z.B. \emph{polymake} oder \emph{jReality}): Durch eine mäßige bis sehr schlechte
Dokumentierung war es teilweise kaum möglich, verlässliche Aussagen über das Verhalten dieser Bibliotheken zu machen. So musste teilweise geraten bzw. intensiv getestet werden, welche Ausgaben erwartet werden können und sogar welche Bedeutung einige Ausgaben überhaupt haben. Diese Probleme traten vor allem bei der Verwendung von \emph{polymake} auf.
Auch bei erfolgreicher Verwendung fällt ein Mangel auf: Die Berechnungsdauer der resultierenden Polytope ist, bei größeren Symmetriegruppen (> 250), sehr langsam. Der Flaschenhals der gesamten Applikation ist im Wesentlichen auf \emph{polymake} zurückzuführen, da hier die meisten Berechnungen relativ lange Zeit beanspruchen. So kann man im Falle von großen Symmetriegruppen den Ausgangspunkt nicht interaktiv via Maus
verschieben und eine flüssige Transformation des Polytops beobachten.

\noindent Beim Projektmanagement gab es zu Anfang des Softwareprojektes schleppende Phasen, in denen fast kein Fortschritt gemacht wurde.
Dies ist wohl auf den anfänglichen Mangel einer Koordinationsperson (eines `Projektleiters`) zurückzuführen; es wurden zwar Anforderungen und
Ideen zur Umsetzen diskutiert, allerdings wurden danach keine konkreten Aufgaben verteilt sodass eine Woche später keine oder kaum Ergebnisse vorlagen. Nach einiger Zeit haben sich aber Personen herauskristallisiert, die die Diskussionsleitung, Protokollführung und Aufgabenverteilung 
in die Hand genommen haben.

Auch einige Verständnisprobleme bei den mathematischen Grundlagen sorgten für einen unerwartet hohen Zeitaufwand: Insbesondere
die Darstellung und Berechnung von Fundamentalbereichen oder auch die korrekte Interpretation von Symmetrietabellen in der Literatur
bereiteten einige Probleme. 

\paragraph{Alternative Lösungen.}
Es wurde diskutiert, die Projektarchitektur nach dem MVC-Pattern zu gestaltet. Wir haben uns aber dennoch für eine ereignisbasierte Architektur
entschieden, da diese mehr Flexibilität im Hinblick auf das Anforderungsmanagement bietet: Sollten bei einem MVC-Projekt schon früh bestimmte
Schnittstellen feststehen, so kann im Ereignisbasierten Paradigma ad-hoc (und vor allem später im Projekt) das Interesse an bestimmten Daten kundgetan werden.

\noindent Auch bei den zum Einsatz kommenden externen Bibliotheken haben wir zwischen Alternativen abgewogen: Zur Visualisierung
sollte zuerst \emph{JavaView}~\cite{javaview} zum Einsatz kommen. Da diese Bibliothek allerdings nur sehr restriktiv (was die Einbettung in ein Java-Programm angeht) verwendet werden kann und zudem schlecht dokumentiert ist, haben wir uns gegen JavaView entschieden.
Für \emph{jReality} sprechen eine bessere Unterstützung für das Einbinden in Java-Applikationen und die open-source-Lizenz: So kann man zur Not,
falls die Dokumentation nicht ausreicht, selber im Code nach schauen, was bestimmte Funktionen tun.

\noindent Für die geometrischen Berechnungen stand ebenfalls ein weiterer Kandidat zur Diskussion: Statt \emph{polymake} wurde der
Einsatz von \emph{CGAL}~\cite{cgal} diskutiert. Obwohl wir der Meinung waren, dass der Umfang, die Funktionalität und die Dokumentation von \emph{CGAL} besser sind, haben wir uns doch gegen den Einsatz entschieden. Der Grund: Die Bibliothek ist in C++ geschrieben und 
verlangt einen erheblichen Mehraufwand beim Anbinden an die Java-Applikation (im Gegensatz zu dem extrem leicht gewichtigen Perl bei \emph{polymake}).

\noindent Bei Java-internen Implementierungsdetails gab es ebenfalls oftmals mehrere (sinnvolle) Alternativen. Hier haben wir uns dann bei den Projekttreffen am Montag nach einer Diskussion jeweils für die passendere Alternative entschieden. Entscheidungen beinhalteten
die Modellierung der Symmetriegruppen (Eine Klasse pro Symmetriegruppe vs. eine Klasse pro Dimension), der Berechnungsanfragen
(Festlegen eines Protokolls für den Dialog mit \emph{polymake} vs. Verschicken von perl-Code) und
Fehlerbehandlung (Exceptions vs. Logging).

\paragraph{Sonstiges.}
Viele allgemeine Erfahrungen wurden während der Projektzeit gemacht. So sind die wöchentlichen Besprechungen unverzichtbar:
Nicht nur weil sich die Gruppe gegenseitig auf den neusten Stand und Lösungsansätze diskutiert, sondern auch um eine gewisse
Gruppendynamik aufrecht zu erhalten. Die Projektarbeit ist viel motivierender, wenn man miteinander Erfahrungen austauscht
und auch über Fehler scherzen kann.

\noindent Auch der Einsatz der Versionsverwaltung \emph{Git} klappte meistens sehr gut. Dank der Versionsgeschichte und der Zuordnung
von Änderungen zu Personen, konnten Fehlerquellen schneller identifiziert werden und mit dem/der Verantwortlichen diskutiert werden.

\noindent Bei der Generierung der Gruppen mit ihren Untergruppen ist es insbesondere aufgefallen, wie sinnvoll Tests für eine schnelle
Prüfung der Korrektheit sind. Auch wenn sich Code im Projekt verändert, kann man sie einfach wieder durchlaufen lassen und so prüfen,
ob noch alles korrekt ist. Tatsächlich wäre es hier wahrscheinlich wirklich sinnvoll gewesen erst die Tests zu schreiben und dann die
Klassen zu implementieren.

\paragraph{Ausblick.}
Es sind noch einige offene Fragen für die Verbesserung des Projektes zu klären:\\
\noindent Wie kann die Performance von \emph{polymake} verbessert werden? Ein naiver Parallelisierungsansatz ist wohl ausgeschlossen,
da die Berechnung innerhalb einer \emph{polymake}-Instanz sequenziell ablaufen und daher genau so lange brauchen.
Wie kann man die GUI intuitiver gestalten, sodass die Bedienung vereinfacht wird?
Können weitere Eingabeparameter angeboten werden, um die angebotenen Berechnungen zu verallgemeinern?



\cleardoublepage
\phantomsection

% Bibliography
\addcontentsline{toc}{section}{Literatur}
\bibliographystyle{alpha}
\bibliography{report}

\end{document}
