\documentclass[ucs,11pt]{beamer}

\usepackage[utf8]{inputenc}
\usepackage[english]{babel}
\usepackage{graphicx}

\include{fu-beamer-template}
\beamersetuncovermixins{\opaqueness<1>{25}}{\opaqueness<2->{15}}

%\titleimage{point-groups}

\title[Punktgruppen]{Visualisierung der 3D/4D Punktgruppen}
\subtitle{Projektvorstellung}
\institute[FU Berlin]{Freie Universität Berlin}
\date[14.11.2013]{14. November 2013}

\begin{document}

\begin{frame}[plain]
	\titlepage
\end{frame}

\begin{frame}{Aufgabe}
	\begin{enumerate}
		\item Modellierung und Darstellung von Symmetrien von 3-dimensionalen Polyedern \pause
		\item und von 4-dimensionalen Polyedern. \pause
		\item Visualisierung als Schlegeldiagramm. \pause
		\item Anzeigen des Fundamentalbereichs.
	\end{enumerate}
\end{frame}


\begin{frame}{Details zu Aufgabe 1 und 2}

\begin{columns}
\begin{column}{.48\textwidth}
\begin{itemize}
	\item Wähle eine Symmetriegruppe
	\item Wähle einen Punkt auf der Oberfläche
 \visible<2->{
	
	\item Abbilden Punktes unter der Symmetriegruppe
}
 \visible<3->{
	\item Bilde die konvexe Hülle der Punktmenge
	\item Darstellung als Schlegeldiagramm 
}
	\end{itemize}

\end{column}%
\hfill%
\begin{column}{.48\textwidth}
\only<1>{
\includegraphics[width=1\textwidth]{javaviewwuerfel3D.png}
}
\only<2>{
\includegraphics[width=1\textwidth]{konvexhull.png}
}
\only<3>{
\includegraphics[width=1\textwidth]{schlegel.png}
}
\end{column}%
\end{columns}
	
\end{frame}

\begin{frame}{Organisation}
		\begin{itemize}
			\item Git, Mailingliste, SplinePad \pause
			\item wöchentliches Meeting \pause
				\begin{itemize}
					\item Wer hat was getan?
					\item Diskussion über aktuelle Probleme
					\item Aufgabenverteilung für die nächste Woche
				\end{itemize} \pause
			\item T-Modell \pause
			\item Java, Eclipse, polymake, JavaView \pause
			\item Coding Style, MIT Lizenz
		\end{itemize}
\end{frame}

\begin{frame}{Architektur}
	\begin{itemize}
		\item Eventbasiert und kein MVC \pause
		\item Pipeline \pause
		\item Brücken zwischen Java und polymake/JavaView \pause
		\item Symmetriegruppen als einzelne Klassen
	\end{itemize}
\end{frame}

\begin{frame}{GUI}
	\begin{itemize}
		\item Wahl zwischen 3D und 4D \pause
		\item Auswahl der Symmetriegruppe und Untergruppe \pause
		\item Darstellung vom Fundamentalbereich zur Punktwahl \pause
		\item Darstellung des Schlegeldiagramms
	\end{itemize}
\end{frame}

\beamersetuncovermixins{\opaqueness<1>{0}}{\opaqueness<2->{1}}
\begin{frame}{Roadmap}
	\begin{itemize}
		\item Bisher:
			\begin{itemize}
				\item Aufgabenstellung verstanden
				\item Organisation umgesetzt
				\item Berechnung der Punktmenge zur Oktaedergruppe in Java mittels Quaternionen
				\item Architekturentwurf
				\item Entwurf der GUI
			\end{itemize} \pause
		\item Nächste Woche:
			\begin{itemize}
				\item Einbindung von polymake und JavaView
				\item Code zur Berechnung der Punktmenge ausbauen
			\end{itemize} \pause
		\item Bis zur nächsten Präsentation:
			\begin{itemize}
				\item Einbinden von polymake und JavaView
				\item 3D-Fall lösen (ohne GUI)
			\end{itemize} \pause
		\item Danach:
			\begin{itemize}
				\item 4D-Fall lösen
				\item GUI implementieren
			\end{itemize}
	\end{itemize}
\end{frame}

\end{document}
