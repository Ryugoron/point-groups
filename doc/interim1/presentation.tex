\documentclass[ucs,11pt]{beamer}

\usepackage[utf8]{inputenc}
\usepackage[english]{babel}
\usepackage{graphicx}

\include{fu-beamer-template}

%\titleimage{point-groups}

\title[Punktgruppen]{Visualisierung der 3D/4D Punktgruppen}
\subtitle{Projektvorstellung}
\institute[FU Berlin]{Freie Universität Berlin}
\date[14.11.2013]{14. November 2013}

\begin{document}

\begin{frame}[plain]
	\titlepage
\end{frame}

\begin{frame}{Aufgabe}
	\begin{enumerate}
	\item Modellierung und Darstellung von Symmetrien von 3-dimensionalen Polyedern. 
	\item Modellierung und Darstellung von Symmetrien von 4-dimensionalen Polyedern. 
	\item Visualisierung als Schlegeldiagramm.
	\item Anzeigen des Fundamentalbereichs.
	\end{enumerate}
\end{frame}


\begin{frame}{Details zu Aufgabe 1 und 2}
	\begin{itemize}
	\item Wähle eine Symmetriegruppe (TODO: Bild vom Würfel)
	\item Wähle einen Punkt auf der Oberfläche
	\item Berechne Punktmenge des Punktes unter der Symmetriegruppe (TODO: Bild)
	\item Bilde die konvexe Hülle der Punktmenge
	\item Darstellung als Schlegeldiagramm (TODO: Bild)	
	\end{itemize}
\end{frame}

\begin{frame}{Organisation}
  	\begin{itemize}
   	 \item Git, Mailingliste, SplinePad
    	\item wöchentliches Meeting
		\begin{itemize}
		\item Wer hat was getan?
		\item Diskussion über aktuelle Probleme
		\item Aufgabenverteilung für die nächste Woche
		\end{itemize}
    	\item T-Prozessmodell
    	\item Java, Ecipse, Polymake, Javaview
	\item Coding Style, MIT Lizenz
  	\end{itemize}
\end{frame}

\begin{frame}{Architektur}
	\begin{itemize}
	\item Eventbasiert und kein MVC
	\item Pipeline
	\item Brücken zwischen Java und Polymake/Javaview
	\item Symmetriegruppen als einzelne Klassen
	\end{itemize}
\end{frame}

\begin{frame}{GUI}
	\begin{itemize}
	\item Wahl zwischen 3D und 4D
	\item Auswahl der Symmetriegruppe und Untergruppe
	\item Darstellung vom Fundamentalbereich zur Punktwahl
	\item Darstellung des Schlegeldiagramms
	\end{itemize}
\end{frame}


\begin{frame}{Roadmap}
  \begin{itemize}
    \item Bisher:
      	\begin{itemize}
        	\item Aufgabenstellung verstanden
	\item Organisation umgesetzt
	\item Berechungen der Punktmenge unter Oktaedergruppe in Java mittels Quaternionen
	\item Architekturentwurf
	\item Entwurf der GUI
     	 \end{itemize}
      \item Nächste Woche:
      	\begin{itemize}
	\item Einbindung von Polymake und Javaview
	\item Code zur Berechung der Punktmenge ausbauen
     	 \end{itemize}
      \item Bis zur nächsten Präsentation:
        	\begin{itemize}
         	 \item Einbinden von Polymake und JavaView
	\item §D-Fall gelösen (ohne GUI)
        	\end{itemize}
      \item Danach:
        	\begin{itemize}
         	 \item 4D-Fall lösen
	\item GUI implementiert
       	 \end{itemize}
  \end{itemize}
\end{frame}
\end{document}

